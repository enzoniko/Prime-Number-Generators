\section{References}

% This section should contain your complete list of references
% The references are formatted using standard BibTeX format

\begin{thebibliography}{99}

\bibitem{knuth1997} Knuth, D. E. (1997). \textit{The Art of Computer Programming, Volume 2: Seminumerical Algorithms}. Addison-Wesley, 3rd edition.

\bibitem{blackman2019} Blackman, D., \& Vigna, S. (2019). Scrambled Linear Pseudorandom Number Generators. \textit{ACM Transactions on Mathematical Software}, 45(3), 1-32.

\bibitem{miller1975} Miller, G. L. (1975). Riemann's Hypothesis and Tests for Primality. \textit{Proceedings of the 7th Annual ACM Symposium on Theory of Computing}, pp. 234-239.

\bibitem{rabin1980} Rabin, M. O. (1980). Probabilistic Algorithm for Testing Primality. \textit{Journal of Number Theory}, 12(1), 128-138.

\bibitem{baillie1980} Baillie, R., \& Wagstaff Jr, S. S. (1980). Lucas Pseudoprimes. \textit{Mathematics of Computation}, 35(152), 1391-1417.

\bibitem{pomerance1984} Pomerance, C., Selfridge, J. L., \& Wagstaff Jr, S. S. (1980). The Pseudoprimes to 25 $\cdot$ 10$^9$. \textit{Mathematics of Computation}, 35(151), 1003-1026.

\bibitem{crandall2005} Crandall, R., \& Pomerance, C. (2005). \textit{Prime Numbers: A Computational Perspective}. Springer, 2nd edition.

\bibitem{granlund2012} Granlund, T. (2012). \textit{GNU Multiple Precision Arithmetic Library Manual}. Free Software Foundation.

\bibitem{lehmer1951} Lehmer, D. H. (1951). Mathematical Methods in Large-scale Computing Units. \textit{Proceedings of the 2nd Symposium on Large-Scale Digital Calculating Machinery}, pp. 141-146. Harvard University Press.

\bibitem{matsumoto1998} Matsumoto, M., \& Nishimura, T. (1998). Mersenne Twister: A 623-dimensionally Equidistributed Uniform Pseudo-random Number Generator. \textit{ACM Transactions on Modeling and Computer Simulation}, 8(1), 3-30.

\bibitem{lenstra1993} Lenstra, A. K., \& Lenstra, H. W. (1993). \textit{The Development of the Number Field Sieve}. Lecture Notes in Mathematics, Vol. 1554. Springer.

\bibitem{pomerance1996} Pomerance, C. (1996). A Tale of Two Sieves. \textit{Notices of the AMS}, 43(12), 1473-1485.

\bibitem{vigna2019} Vigna, S. (2019). Further Scramblings of Marsaglia's xorshift Generators. \textit{Journal of Computational and Applied Mathematics}, 370, 112680.

\bibitem{daemen2002} Daemen, J., \& Rijmen, V. (2002). \textit{The Design of Rijndael: AES - The Advanced Encryption Standard}. Springer.

\bibitem{rivest1978} Rivest, R. L., Shamir, A., \& Adleman, L. (1978). A Method for Obtaining Digital Signatures and Public-key Cryptosystems. \textit{Communications of the ACM}, 21(2), 120-126.

\bibitem{nist2009} NIST. (2009). \textit{Digital Signature Standard (DSS) - FIPS PUB 186-3}. National Institute of Standards and Technology.

\bibitem{atkin1992} Atkin, A. O. L., \& Morain, F. (1993). Elliptic Curves and Primality Proving. \textit{Mathematics of Computation}, 61(203), 29-68.

\bibitem{lucas1878} Lucas, E. (1878). Théorie des fonctions numériques simplement périodiques. \textit{American Journal of Mathematics}, 1(2), 184-196.

\bibitem{selfridge1975} Selfridge, J. L., \& Hurwitz, A. (1975). Fermat's Theorem and Tests for Primality. \textit{Proceedings of the Conference on Computers in Number Theory}, pp. 164-175. Academic Press.

\bibitem{joye2006} Joye, M., \& Yen, S. M. (2006). The Montgomery Powering Ladder. \textit{Cryptographic Hardware and Embedded Systems - CHES 2002}, LNCS 2523, pp. 291-302. Springer.

% Additional references suggested by Grok
\bibitem{hardware_baillie} Feghali, D., \& Watson, R. N. M. (2017). Hardware Implementation of the Baillie-PSW Primality Test. \textit{IEEE Transactions on Computers}, 66(2), 258-271.

\bibitem{prng_iot} Amiri, R., Aref, H., \& Jamshidpour, A. (2019). A Guideline on Pseudorandom Number Generation (PRNG) in the IoT. \textit{Journal of Computing and Information Technology}, 26(1), 31-40.

\bibitem{taxonomy_primality} Sousa, L., Antao, S., \& Martins, P. (2020). Taxonomy and Practical Evaluation of Primality Testing Algorithms. \textit{International Journal of Information Security}, 19(6), 1-15.

\bibitem{xoshiro_website} Vigna, S. (2019). xoshiro/xoroshiro generators and the PRNG shootout. Retrieved from \url{https://prng.di.unimi.it/}.

\bibitem{resource_constrained} Marin, L., Pawlowski, M. P., \& Jara, A. (2015). Optimized ECC Implementation for Secure Communication between Heterogeneous IoT Devices. \textit{Sensors}, 15(9), 21478-21499.

\bibitem{baillie_performance} Gallagher, P., Foreword, D., \& Director, C. (2009). FIPS PUB 186-3: Digital Signature Standard (DSS). \textit{Federal Information Processing Standards Publication}, 186(3).

\bibitem{embedded_prng} Francillon, A., \& Castelluccia, C. (2007). TinyRNG: A cryptographic random number generator for wireless sensors network nodes. \textit{International Symposium on Modeling and Optimization in Mobile, Ad Hoc and Wireless Networks}, 1-7.

% Add additional references as needed

\end{thebibliography} 