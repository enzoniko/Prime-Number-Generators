\section{Conclusion}

This project explored the implementation and performance of pseudo-random number generation (PRNG) and primality testing algorithms, crucial for cryptography, particularly in resource-constrained environments \cite{resource_constrained}. Our analysis provides insights into algorithm behavior, efficiency, and practical application.

\subsection{Summary of Contributions}

Key contributions include the implementation of LCG and Xoshiro256++ PRNGs and Miller-Rabin and Baillie-PSW primality tests, all supporting numbers up to 4096 bits and optimized for resource constraints \cite{energy_efficient}. We developed a comprehensive benchmarking framework \cite{embedded_benchmarking}, performed detailed performance analysis across bit lengths, and created a modular, documented C++ codebase suitable for IoT and embedded systems \cite{iot_survey}.

\subsection{Key Findings}

Our experiments yielded several key findings:

\begin{itemize}
    \item \textbf{PRNGs}: While LCG is faster, Xoshiro256++ offers superior statistical quality suitable for cryptographic use, whereas LCG fails basic randomness tests and is inappropriate for secure applications.
    \item \textbf{Primality Tests}: Baillie-PSW provides high reliability and predictable performance, ideal for high-assurance needs. Miller-Rabin offers speed flexibility via adjustable rounds but can become computationally intensive for large numbers and high certainty.
\end{itemize}

\subsection{Practical Implications}

The findings offer practical guidance for selecting algorithms in cryptographic systems. For instance, in IoT devices, the choice involves balancing Xoshiro256++'s quality against LCG's speed, and Baillie-PSW's reliability against Miller-Rabin's potential speed \cite{energy_prng, prime_iot}. Statistical analysis confirmed LCG is only suitable for non-critical tasks \cite{lcg_applications, nist_test_suite}. Baillie-PSW's strength makes it ideal for high-security key generation \cite{baillie_attacks, pomerance2001}. Energy efficiency considerations, particularly for battery-powered devices, might favor fewer Miller-Rabin rounds for initial screening \cite{iot_survey}. The implementation's modular design facilitates integration into various embedded systems \cite{embedded_benchmarking, embedded_crypto}.

\subsection{Future Work}

Potential future work includes:

\begin{itemize}
    \item Extending the library with more cryptographic primitives (e.g., hashing, symmetric encryption).
    \item Integrating the library into larger frameworks or applications.
    \item Investigating and potentially mitigating side-channel vulnerabilities.
    \item Conducting performance analysis on actual resource-constrained hardware.
    \item Exploring alternative large number libraries or custom arithmetic routines.
\end{itemize}

\subsection{Final Remarks}

This project delivers a practical implementation and comparative analysis of fundamental cryptographic algorithms, emphasizing performance in resource-constrained settings \cite{resource_constrained, embedded_crypto}. The results offer valuable insights for designing future cryptographic systems, especially those operating under significant resource limitations. 