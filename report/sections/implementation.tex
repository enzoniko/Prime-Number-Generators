\section{Implementation}

This section details the implementation of the random number generation and primality testing algorithms, with special attention to optimizations for resource-constrained environments \cite{energy_efficient, iot_survey}.

\subsection{Random Number Generation Algorithms}

\subsubsection{Linear Congruential Generator (LCG)}
The Linear Congruential Generator is one of the simplest PRNGs, defined by the recurrence relation:

\begin{equation}
X_{n+1} = (a \cdot X_n + c) \bmod m
\end{equation}

Where:
\begin{itemize}
    \item $X_n$ is the current state
    \item $X_{n+1}$ is the next state
    \item $a$ is the multiplier
    \item $c$ is the increment
    \item $m$ is the modulus
\end{itemize}

Our implementation uses the parameters from the Numerical Recipes version of the LCG:
\begin{itemize}
    \item $a = 1664525$
    \item $c = 1013904223$
    \item $m = 2^{32}$
\end{itemize}

These parameters were chosen for efficient implementation on 32-bit systems while providing a reasonable period. The implementation leverages bitwise operations to optimize the modulo operation, which is crucial for performance in resource-constrained environments \cite{energy_efficient}.

\subsubsection{Xoshiro256++}
Xoshiro256++ is a modern, fast PRNG with excellent statistical properties, making it suitable for cryptographic applications that don't require cryptographic security. The algorithm operates on a 256-bit state divided into four 64-bit integers.

The core algorithm is defined as:

\begin{verbatim}
uint64_t xoshiro256pp(void) {
    uint64_t result = rotl(s[0] + s[3], 23) + s[0];
    uint64_t t = s[1] << 17;
    s[2] ^= s[0];
    s[3] ^= s[1];
    s[1] ^= s[2];
    s[0] ^= s[3];
    s[2] ^= t;
    s[3] = rotl(s[3], 45);
    return result;
}
\end{verbatim}

Our implementation includes optimizations for memory usage and computational efficiency, which are critical for resource-constrained environments \cite{embedded_benchmarking}:
\begin{itemize}
    \item Efficient bit manipulation operations to minimize instruction count
    \item Careful state management to avoid unnecessary memory operations
    \item Unrolled loops where beneficial for performance
\end{itemize}

\subsection{Primality Testing Algorithms}

\subsubsection{Miller-Rabin Primality Test}
The Miller-Rabin primality test is a probabilistic algorithm that determines whether a given number is prime. The test is based on an extension of Fermat's little theorem and can be tuned to provide different levels of certainty.

The algorithm works as follows:
\begin{enumerate}
    \item Express $n - 1$ as $2^r \cdot d$ where $d$ is odd
    \item Choose a random base $a$ in the range $[2, n-2]$
    \item Compute $x = a^d \bmod n$
    \item If $x = 1$ or $x = n-1$, the test passes for this base
    \item For $i = 1$ to $r-1$:
    \begin{itemize}
        \item Compute $x = x^2 \bmod n$
        \item If $x = n-1$, the test passes for this base
        \item If $x = 1$, return composite
    \end{itemize}
    \item If we reach this point, return composite
    \item Repeat steps 2-6 for multiple bases to reduce the probability of error
\end{enumerate}

Our implementation includes the following optimizations for resource-constrained environments \cite{energy_efficient}:
\begin{itemize}
    \item Efficient modular exponentiation using the square-and-multiply algorithm
    \item Early termination when composite status is detected
    \item Memory-efficient representation of large integers
    \item Dynamic adjustment of the number of test rounds based on the desired certainty level and available resources
\end{itemize}

\subsubsection{Baillie-PSW Primality Test}
The Baillie-PSW test combines the Miller-Rabin test with a Lucas pseudoprime test, providing a more robust primality test with no known counterexamples for properly implemented versions.

The algorithm consists of the following steps:
\begin{enumerate}
    \item Perform trial division by small primes (2, 3, 5, 7, 11, 13, 17, 19, 23)
    \item Perform a single Miller-Rabin test with base 2
    \item Perform a Lucas pseudoprime test
\end{enumerate}

For the Lucas test:
\begin{enumerate}
    \item Find the first $D$ in the sequence 5, -7, 9, -11, 13, -15, ... such that the Jacobi symbol $(D/n) = -1$
    \item Calculate Lucas sequences $U_k$ and $V_k$ with parameters $P=1$ and $Q=(1-D)/4$
    \item Compute $U_{n+1}$ and check if $U_{n+1} \equiv 0 \pmod{n}$
\end{enumerate}

Our implementation includes several optimizations specific to resource-constrained environments \cite{embedded_benchmarking, energy_efficient}:
\begin{itemize}
    \item Efficient computation of the Jacobi symbol
    \item Optimized Lucas sequence computation using modular arithmetic
    \item Memory-efficient state representation
    \item Combined implementation that shares code with the Miller-Rabin test where possible
\end{itemize}

\subsection{Implementation Considerations}

\subsubsection{Memory Optimization}
In resource-constrained environments, memory usage is often a critical limitation. Our implementations include the following memory optimizations \cite{iot_survey}:
\begin{itemize}
    \item In-place operations where possible to avoid temporary allocations
    \item Reuse of memory buffers across different algorithm phases
    \item Careful management of stack usage to avoid stack overflow in embedded environments
    \item Compile-time configuration options to trade memory usage for performance based on the target environment
\end{itemize}

\subsubsection{Computational Efficiency}
To optimize computational efficiency, our implementations employ several techniques \cite{energy_efficient}:
\begin{itemize}
    \item Bitwise operations instead of more expensive arithmetic when possible
    \item Loop unrolling for performance-critical sections
    \item Strength reduction (replacing expensive operations with equivalent but less expensive ones)
    \item Early termination of algorithms when definitive results are available
    \item Cache-friendly memory access patterns
\end{itemize}

\subsubsection{Energy Efficiency}
Energy consumption is particularly important in battery-powered IoT devices. Our implementations include the following energy-saving optimizations \cite{energy_efficient}:
\begin{itemize}
    \item Instruction count minimization to reduce CPU active time
    \item Algorithm adaptation to minimize high-energy operations (e.g., memory accesses, floating-point operations)
    \item Configuration options to trade performance for energy efficiency
    \item Support for processor sleep modes between operations when appropriate
\end{itemize}

\subsubsection{Portability}
Our implementations are designed to be portable across different resource-constrained platforms \cite{embedded_benchmarking}:
\begin{itemize}
    \item Standard C implementation with minimal dependency on external libraries
    \item Abstraction layers for platform-specific optimizations
    \item Consistent behavior across 8-bit, 16-bit, 32-bit, and 64-bit architectures
    \item Compile-time configuration options for different environments
\end{itemize}

\subsection{Implementation Validation}
To ensure correctness, our implementations have been validated through:
\begin{itemize}
    \item Testing against known test vectors from the NIST Statistical Test Suite \cite{nist_test_suite}
    \item Comparison with reference implementations
    \item Formal verification of critical algorithm components
    \item Extensive testing across different input sizes and edge cases
\end{itemize}

This comprehensive implementation approach ensures that our algorithms are suitable for deployment in a wide range of resource-constrained environments, from IoT devices to embedded systems, while maintaining the necessary security properties for cryptographic applications. 