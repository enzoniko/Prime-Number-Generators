\section{Pseudo-Random Number Generators (PRNGs)}

This section covers the pseudo-random number generators (PRNGs) implemented in this project. We have chosen to implement and analyze two algorithms: the Linear Congruential Generator (LCG) and the Xoshiro256++ generator. Both algorithms have been implemented with the capability to generate random numbers up to 4096 bits in length, with particular focus on their performance characteristics in resource-constrained environments.

\subsection{Theoretical Background and Selection Criteria}

Pseudo-random number generators are deterministic algorithms that produce sequences of numbers that approximate the properties of random numbers. These algorithms typically start with an initial value called a seed and apply transformations to generate subsequent values. The quality of a PRNG is assessed through various statistical measures, as well as practical considerations regarding implementation complexity and computational efficiency.

In cryptographic applications, especially those deployed in resource-constrained systems, PRNGs must satisfy several properties \cite{prng_iot}:

\begin{itemize}
    \item \textbf{Uniform Distribution}: The generated numbers should be uniformly distributed across their range, ensuring unbiased sampling.
    
    \item \textbf{Independence}: Each generated number should be statistically independent of previous numbers, preventing predictability patterns.
    
    \item \textbf{Long Period}: The sequence should repeat only after a very large number of generations, particularly critical for applications requiring extensive random sampling.
    
    \item \textbf{Unpredictability}: Given a sequence of previously generated numbers, it should be computationally infeasible to predict the next number, a vital property for security applications.
    
    \item \textbf{Statistical Randomness}: Output sequences must pass standard statistical tests for randomness (e.g., NIST STS \cite{nist_test_suite}).
    
    \item \textbf{Efficiency}: Generation must be fast and consume minimal resources (CPU cycles, memory), a crucial consideration for systems with limited resources \cite{embedded_prng}.
\end{itemize}

While the PRNGs implemented in this project are not cryptographically secure in the strictest sense, they provide a foundation for understanding and implementing more secure generators. Our selection of LCG and Xoshiro256++ was informed by their demonstrated efficiency in resource-constrained environments, as documented in the literature \cite{prng_iot, xoshiro_website}.

\subsection{Linear Congruential Generator (LCG)}

\subsubsection{Algorithm Description and Justification}

The Linear Congruential Generator is one of the oldest and simplest PRNGs. Despite its simplicity, it remains relevant in resource-constrained environments due to its minimal computational and memory requirements \cite{prng_iot}. It operates based on the following recurrence relation:

\begin{equation}
X_{n+1} = (a \cdot X_n + c) \bmod m
\end{equation}

where:
\begin{itemize}
    \item $X_n$ is the sequence of generated values
    \item $a$ is the multiplier
    \item $c$ is the increment
    \item $m$ is the modulus
    \item $X_0$ is the initial seed
\end{itemize}

In our implementation, we use the following parameters:
\begin{itemize}
    \item $a = 6364136223846793005$ (from the POSIX standard)
    \item $c = 1$
    \item $m = 2^{64}$ (implicit due to 64-bit integer overflow)
\end{itemize}

The primary advantages of LCG in resource-constrained environments include:

\begin{itemize}
    \item \textbf{Simplicity}: Very easy to implement and requires minimal state.
    \item \textbf{Speed}: Extremely fast due to simple arithmetic operations.
    \item \textbf{Low Memory Footprint}: Requires storing only the current state (one integer), ideal for systems with severe memory constraints.
    \item \textbf{Computational Efficiency}: The simplicity of operations translates to lower computational cost.
\end{itemize}

For generating multi-precision numbers, we combine multiple 64-bit outputs from the LCG to create numbers of arbitrary bit lengths, an approach that maintains the algorithm's efficiency while extending its capability to generate large numbers.

\subsubsection{Implementation Details}

Our LCG implementation follows an object-oriented approach with a clean interface:

\begin{lstlisting}[language=C++, caption=LCG Implementation (Header)]
class LCG : public PRNGInterface {
private:
    uint64_t state;  // Current state of the generator
    
    // LCG parameters - using values from the POSIX standard
    static const uint64_t LCG_A = 6364136223846793005ULL;
    static const uint64_t LCG_C = 1ULL;
    // Modulus M is implicitly 2^64 due to uint64_t overflow

public:
    /**
     * @brief Construct a new LCG with a seed
     * 
     * @param seed Initial seed value, or 0 for automatic seeding
     */
    LCG(uint64_t seed = 0) : state(seed) {
        if (seed == 0) {
            // Use system clock as seed if none provided
            state = std::chrono::high_resolution_clock::now().time_since_epoch().count();
            if (state == 0) state = 1; // Avoid state being 0
        }
    }
    
    /**
     * @brief Generate the next 64-bit pseudo-random value
     * 
     * @return uint64_t The next random value
     */
    uint64_t next_u64() override {
        state = LCG_A * state + LCG_C; // Modulo 2^64 is implicit due to overflow
        return state;
    }
    
    /**
     * @brief Generate a k-bit random number
     * 
     * @param result GMP integer to store the result
     * @param k Number of bits
     */
    void randbits(mpz_t result, int k) override {
        if (k <= 0) {
            mpz_set_ui(result, 0);
            return;
        }

        int bits_per_output = 64;
        int num_outputs = std::ceil(static_cast<double>(k) / bits_per_output);

        mpz_set_ui(result, 0); // Initialize GMP integer to 0

        for (int i = 0; i < num_outputs; ++i) {
            uint64_t rand_val = next_u64();
            mpz_mul_2exp(result, result, bits_per_output); // Left shift by 64 bits
            mpz_add_ui(result, result, rand_val); // Add the new 64 bits
        }

        // Trim excess bits if necessary
        int bits_generated = num_outputs * bits_per_output;
        if (bits_generated > k) {
            mpz_fdiv_q_2exp(result, result, bits_generated - k); // Right shift
        }

        // Ensure the number has exactly k bits (set MSB to 1)
        if (k > 0) {
            mpz_setbit(result, k - 1);
        }
    }
};
\end{lstlisting}

\subsubsection{Analysis of Generated Results}

The performance results, including the time required to generate random numbers of different bit sizes using both PRNGs, as well as metrics related to memory usage and energy consumption where applicable, are presented in the Results section (\autoref{sec:results}).

\subsection{Xoshiro256++ Generator}

\subsubsection{Algorithm Description and Justification}

The Xoshiro256++ generator represents the current state of the art in non-cryptographic PRNGs, developed by Blackman and Vigna \cite{blackman2019}. It offers an excellent balance between statistical quality, speed, and state size, making it well-suited for resource-constrained systems while providing superior randomness properties compared to simpler generators like LCG \cite{xoshiro_website}.

The algorithm maintains a state of 256 bits, represented as four 64-bit integers. While this is larger than the LCG's state, it remains modest in comparison to other high-quality PRNGs such as the Mersenne Twister (which requires 2.5KB of state) \cite{matsumoto1998}, making it suitable for memory-constrained environments.

The state transition function is defined as:

\begin{align}
t &= s[1] << 17 \\
s[2] &= s[2] \oplus s[0] \\
s[3] &= s[3] \oplus s[1] \\
s[1] &= s[1] \oplus s[2] \\
s[0] &= s[0] \oplus s[3] \\
s[2] &= s[2] \oplus t \\
s[3] &= \text{rotl}(s[3], 45)
\end{align}

The output function for Xoshiro256++ is:
\begin{equation}
\text{output} = \text{rotl}(s[0] + s[3], 23) + s[0]
\end{equation}

where $\text{rotl}(x, k)$ is a bitwise left rotation of $x$ by $k$ bits.

The key advantages of Xoshiro256++ for resource-constrained systems include:

\begin{itemize}
    \item \textbf{Exceptional Statistical Properties}: Xoshiro256++ passes all statistical tests in the TestU01 suite's BigCrush battery, ensuring high-quality randomness \cite{blackman2019, xoshiro_website}.
    
    \item \textbf{Long Period}: The generator has a period of $2^{256} - 1$, making it suitable for applications requiring extensive random sampling without repetition.
    
    \item \textbf{Computational Efficiency}: Despite its sophistication, Xoshiro256++ is extremely fast, with benchmarks showing it achieves up to 2560 MB/s on ARM Cortex-A53 processors commonly found in IoT and embedded devices \cite{xoshiro_website}.
    
    \item \textbf{Moderate Memory Footprint}: Its 256-bit state (32 bytes) is compact enough for memory-constrained devices while providing sufficient complexity for high-quality randomness.
    
    \item \textbf{Energy Efficiency}: The algorithm uses simple bitwise operations (XOR, shifts, rotations) that are energy-efficient on most hardware architectures, making it suitable for battery-powered devices \cite{prng_iot}.
\end{itemize}

\subsubsection{Implementation Details}

Our implementation of Xoshiro256++ follows the object-oriented approach with a similar interface to the LCG:

\begin{lstlisting}[language=C++, caption=Xoshiro256++ Implementation (Header)]
/**
 * @brief Helper function for 64-bit left rotation
 * 
 * @param x Value to rotate
 * @param k Number of bits to rotate by
 * @return uint64_t Rotated value
 */
static inline uint64_t rotl(const uint64_t x, int k) {
    return (x << k) | (x >> (64 - k));
}

/**
 * @brief Xoshiro256++ pseudo-random number generator
 * 
 * A fast, high-quality PRNG with a state size of 256 bits.
 * 
 * Reference: Blackman, D., & Vigna, S. (2019). Scrambled Linear Pseudorandom 
 * Number Generators. arXiv preprint arXiv:1805.01407v5.
 * 
 * Official implementation: https://prng.di.unimi.it/
 */
class Xoshiro256pp : public PRNGInterface {
private:
    uint64_t s[4]; // 256 bits of state
    
    /**
     * @brief Seed the state array
     * 
     * @param seed_val Initial seed value
     */
    void seed_state(uint64_t seed_val) {
         std::mt19937_64 seeder(seed_val);
         // Fill the state array with values from std::mt19937_64
         for (int i = 0; i < 4; ++i) {
             s[i] = seeder();
         }
         
         // Ensure the initial state is not all zeros
         bool all_zeros = true;
         for (int i = 0; i < 4; ++i) {
             if (s[i] != 0) {
                 all_zeros = false;
                 break;
             }
         }
         if (all_zeros) {
             s[0] = 1; // Set a non-zero state
         }
    }

public:
    /**
     * @brief Construct a new Xoshiro256++ generator
     * 
     * @param seed Initial seed value, or 0 for automatic seeding
     */
    Xoshiro256pp(uint64_t seed = 0) {
        if (seed == 0) {
            // Use system clock as seed if none provided
            seed = std::chrono::high_resolution_clock::now().time_since_epoch().count();
        }
        seed_state(seed);
    }
    
    /**
     * @brief Generate the next 64-bit pseudo-random value
     * 
     * @return uint64_t The next random value
     */
    uint64_t next_u64() override {
        // The core Xoshiro256++ algorithm:
        const uint64_t result = rotl(s[0] + s[3], 23) + s[0]; // ++ scrambler part

        const uint64_t t = s[1] << 17;

        s[2] ^= s[0];
        s[3] ^= s[1];
        s[1] ^= s[2];
        s[0] ^= s[3];

        s[2] ^= t;

        s[3] = rotl(s[3], 45);

        return result;
    }
    
    /**
     * @brief Generate a k-bit random number
     * 
     * @param result GMP integer to store the result
     * @param k Number of bits
     */
    void randbits(mpz_t result, int k) override {
        if (k <= 0) {
            mpz_set_ui(result, 0);
            return;
        }

        int bits_per_output = 64;
        int num_outputs = std::ceil(static_cast<double>(k) / bits_per_output);

        mpz_set_ui(result, 0); // Initialize GMP integer to 0

        for (int i = 0; i < num_outputs; ++i) {
            uint64_t rand_val = next_u64();
            mpz_mul_2exp(result, result, bits_per_output); // Left shift by 64 bits
            mpz_add_ui(result, result, rand_val); // Add the new 64 bits
        }

        // Trim excess bits if necessary
        int bits_generated = num_outputs * bits_per_output;
        if (bits_generated > k) {
            mpz_fdiv_q_2exp(result, result, bits_generated - k); // Right shift
        }

        // Ensure the number has exactly k bits (set MSB to 1)
        if (k > 0) {
            mpz_setbit(result, k - 1);
        }
    }
};
\end{lstlisting}

\subsubsection{Analysis of Generated Results}

The performance results, including the time required to generate random numbers of different bit sizes using both PRNGs, as well as metrics related to memory usage and energy consumption where applicable, are presented in the Results section (\autoref{sec:results}).

\subsection{Implementation of Arbitrary-Precision Random Number Generation}

For both PRNGs, generating numbers of arbitrary precision (up to 4096 bits) requires additional handling beyond what the basic algorithms provide. We use the GMP library for this purpose \cite{granlund2012}:

\begin{enumerate}
    \item We first determine how many 64-bit blocks are needed to represent a number of the desired bit length.
    
    \item We call the PRNG's `next()` method repeatedly to fill these blocks.
    
    \item We use GMP's "mpz\_import" function to convert the array of 64-bit blocks into a single arbitrary-precision integer.
    
    \item We ensure the result has exactly the requested number of bits by setting the most significant bit and masking off any excess bits.
\end{enumerate}

This approach allows us to generate uniformly distributed random numbers of any bit length up to the capacity of the system's memory, maintaining efficiency while extending the capability to large numbers required for cryptographic applications.

\subsection{Performance Considerations in Resource-Constrained Environments}

The two PRNGs have different performance characteristics, particularly relevant when deployed in resource-constrained systems:

\begin{itemize}
    \item \textbf{LCG}: Is simpler and requires less state (64 bits vs. 256 bits), resulting in a smaller memory footprint. It performs fewer operations per generation, potentially offering better energy efficiency for extremely constrained devices. However, it has known statistical weaknesses, especially in the lower bits \cite{knuth1997}. These weaknesses are partially mitigated in our implementation by using only the higher-quality bits and combining multiple outputs for large numbers.
    
    \item \textbf{Xoshiro256++}: Offers superior statistical properties and a much longer period, making it more suitable for applications requiring high-quality randomness. While it requires slightly more memory and computational resources than LCG, it remains highly efficient compared to other high-quality PRNGs. According to benchmarks by Vigna \cite{xoshiro_website}, Xoshiro256++ demonstrates excellent performance on ARM architectures common in IoT devices, with generation speeds comparable to simpler generators.
\end{itemize}

For resource-constrained systems, the choice between these generators involves a trade-off:

\begin{itemize}
    \item For extremely limited devices where every byte of memory and cycle of CPU matters, LCG may be preferable due to its minimal footprint.
    
    \item For devices with slightly more resources where randomness quality is important, Xoshiro256++ offers a better balance of quality and efficiency \cite{prng_iot}.
\end{itemize}

The detailed timing comparisons for generating random numbers of various bit lengths are presented in the Results section (\autoref{sec:results}).

\subsection{Code Snippets and Implementation Optimizations}

Below are key sections of the implementation for both generators, highlighting optimizations for resource-constrained environments:

\begin{lstlisting}[language=C++, caption=LCG Constructor Implementation]
/**
 * @brief Construct a new LCG with a seed
 * 
 * @param seed Initial seed value, or 0 for automatic seeding
 */
LCG(uint64_t seed = 0) : state(seed) {
    if (seed == 0) {
        // Use system clock as seed if none provided
        state = std::chrono::high_resolution_clock::now().time_since_epoch().count();
        if (state == 0) state = 1; // Avoid state being 0
    }
}
\end{lstlisting}

\begin{lstlisting}[language=C++, caption=Xoshiro256++ Seeding Implementation]
/**
 * @brief Seed the state array
 * 
 * @param seed_val Initial seed value
 */
void seed_state(uint64_t seed_val) {
     std::mt19937_64 seeder(seed_val);
     // Fill the state array with values from std::mt19937_64
     for (int i = 0; i < 4; ++i) {
         s[i] = seeder();
     }
     
     // Ensure the initial state is not all zeros
     bool all_zeros = true;
     for (int i = 0; i < 4; ++i) {
         if (s[i] != 0) {
             all_zeros = false;
             break;
         }
     }
     if (all_zeros) {
         s[0] = 1; // Set a non-zero state
     }
}
\end{lstlisting}

Both seeding implementations incorporate techniques to enhance the quality of initialization, ensuring good randomness even with simple seed values—a consideration particularly important in embedded systems where high-quality entropy sources may be limited \cite{embedded_prng}.

\subsection{Performance Comparison}
We benchmarked the implemented LCG and Xoshiro256++ algorithms based on the methodology described earlier. Key metrics include execution time for generating a fixed number of random bits and estimated memory footprint. The detailed timing comparisons for generating random numbers of various bit lengths are presented in the Results section (\autoref{sec:results}).

\subsection{Conclusion}
Performance was evaluated based on execution speed and memory usage. The detailed timing comparisons for generating random numbers of various bit lengths are presented in the Results section (\autoref{sec:results}). 

Having established our generators and benchmarked their performance up to 4096 bits, we now feed these bit-strings into probabilistic primality tests to measure verification costs.