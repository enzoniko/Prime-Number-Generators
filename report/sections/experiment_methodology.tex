\section{Experiment Methodology}

This section details our rigorous methodology for evaluating random number generation and primality testing algorithms in resource-constrained environments. The approach follows established principles for benchmarking embedded systems \cite{resource_constrained, embedded_benchmarking} and emphasizes reproducibility and statistical validity.

\subsection{Experimental Design}
Our methodology adheres to the following principles:
\begin{itemize}
    \item \textbf{Reproducibility}: All experiments are designed to be fully reproducible by other researchers.
    \item \textbf{Statistical Validity}: Multiple runs with statistical analysis of results ensure significance.
    \item \textbf{Controlled Variables}: We isolate the impact of individual parameters by controlling all other variables.
    \item \textbf{Resource Monitoring}: Comprehensive monitoring of CPU, memory, and energy consumption provides insights into resource utilization patterns relevant to constrained environments \cite{energy_efficient}.
    \item \textbf{Scaling Analysis}: We analyze how algorithms scale with different input sizes and resource restrictions.
\end{itemize}

\subsection{Test Environment}
\begin{itemize}
    \item \textbf{Hardware Configuration}: [Specify processor model, clock speed, memory capacity, cache sizes]
    \item \textbf{Software Environment}: [Operating system, compiler version, optimization flags]
    \item \textbf{Resource Constraints}: We simulate various resource constraints typical of embedded and IoT systems \cite{iot_survey}, including:
    \begin{itemize}
        \item Limited memory availability (8MB, 16MB, 32MB)
        \item Restricted CPU power (single core, limited clock speed)
        \item Energy constraints (measured in joules per operation)
    \end{itemize}
\end{itemize}

\subsection{Measurement Methodology}
We measure the following performance metrics:
\begin{itemize}
    \item \textbf{Execution Time}: Wall-clock time required to complete operations
    \item \textbf{Memory Usage}: Peak and average memory consumption during execution
    \item \textbf{Energy Consumption}: Power used during algorithm execution
    \item \textbf{CPU Utilization}: Percentage of CPU resources required
    \item \textbf{Scaling Behavior}: How performance changes with varying input sizes and resource constraints
\end{itemize}

\subsubsection{Timing Procedure}
To ensure accurate time measurements:
\begin{itemize}
    \item Perform warm-up runs to eliminate cold-cache effects
    \item Execute each experiment for a minimum of 30 iterations
    \item Remove outliers beyond 2 standard deviations
    \item Calculate mean, median, standard deviation, and confidence intervals (95\%)
    \item Report minimum and maximum observed execution times
\end{itemize}

\subsection{Experimental Scenarios}
\subsubsection{Random Number Generation}
For each random number generation algorithm:
\begin{itemize}
    \item Generate sequences of varying lengths (10³, 10⁴, 10⁵, 10⁶)
    \item Apply NIST Statistical Test Suite components \cite{nist_test_suite} to evaluate randomness quality
    \item Measure execution time, memory usage, and energy consumption
    \item Analyze the trade-off between randomness quality and resource usage
\end{itemize}

\subsubsection{Primality Testing}
For each primality testing algorithm:
\begin{itemize}
    \item Test with numbers of varying bit lengths (32, 64, 128, 256, 512, 1024 bits)
    \item Include both prime and composite test cases
    \item Measure execution time, memory footprint, and energy usage
    \item Evaluate accuracy vs. resource usage trade-offs at different certainty levels
\end{itemize}

\subsection{Data Collection and Analysis}
\begin{itemize}
    \item \textbf{Automated Logging}: All experimental data is automatically logged to prevent human error
    \item \textbf{Statistical Analysis}: We apply rigorous statistical methods to identify significant differences between algorithms
    \item \textbf{Visualization}: Results are presented through performance profiles, scaling graphs, and resource utilization charts
\end{itemize}

\subsection{Validation Strategies}
To ensure correctness of our experimental results:
\begin{itemize}
    \item Cross-validate with established benchmark results where available
    \item Verify random number generator output against NIST statistical test suite recommendations \cite{nist_test_suite}
    \item Confirm primality test results against numbers with known primality status
    \item Conduct sensitivity analysis to identify potential sources of experimental error
\end{itemize}

\subsection{Documentation}
We document the following to ensure reproducibility:
\begin{itemize}
    \item Complete experimental setup with hardware and software specifications
    \item Algorithm implementations with version control references
    \item Raw experimental data and analysis scripts
    \item Environmental variables and system configuration
\end{itemize}

\subsection{Limitations and Assumptions}
We acknowledge the following limitations in our methodology:
\begin{itemize}
    \item Laboratory conditions may differ from real-world deployment scenarios
    \item Hardware-specific optimizations may not translate to all platforms
    \item Our resource constraint simulations approximate but may not perfectly match all target environments
\end{itemize}

This methodology follows established benchmarking practices for embedded and resource-constrained systems \cite{resource_constrained, embedded_benchmarking, energy_efficient} while adapting them to the specific requirements of cryptographic primitives evaluation. 