\section{Repository Structure and Implementation}

The project has been implemented as a structured C++ codebase with a focus on modularity, extensibility, and performance. This section provides an overview of the repository organization and the key components of the implementation.

\subsection{Directory Organization}

The repository follows a typical C++ project structure with clear separation of concerns:

\begin{lstlisting}[language=bash, caption=Project directory structure]
.
├── src/                # Source files
│   ├── prng/           # Pseudo-Random Number Generator implementations
│   ├── primality/      # Primality testing algorithm implementations
│   ├── utils/          # Utility functions
│   ├── tests/          # Test implementations
│   └── experiments/    # Performance measurement programs
├── include/            # Header files
│   ├── prng/           # PRNG headers
│   ├── primality/      # Primality test headers
│   └── utils/          # Utility headers
├── docs/               # Documentation
│   └── diagrams/       # PlantUML diagrams
├── experiments/        # RISC-V performance and energy measurement tools
│   ├── bin/            # Compiled experiment binaries
│   ├── configs/        # Configuration files for experiments
│   ├── results/        # Experiment results
│   └── scripts/        # Experiment execution scripts
├── Makefile            # Build system
├── README.md           # Project overview
└── results/            # Performance results and analysis
\end{lstlisting}

\subsection{Key Components}

\subsubsection{Source Code Organization}

The source code is logically partitioned based on functionality:

\begin{itemize}
    \item \textbf{PRNG Module}: Contains implementations of the Linear Congruential Generator (LCG) and Xoshiro256++ pseudo-random number generators.
    
    \item \textbf{Primality Module}: Contains implementations of the Miller-Rabin and Baillie-PSW primality testing algorithms.
    
    \item \textbf{Utilities}: Provides common functionality used across the project, such as bit manipulation, timing functions, and GMP wrappers.
    
    \item \textbf{Tests}: Contains unit tests and validation code to ensure the correctness of the implementations.
    
    \item \textbf{Experiments}: Contains specialized programs for measuring and benchmarking algorithm performance.
\end{itemize}

\subsubsection{Dependency Management}

The project depends on the following external libraries:

\begin{itemize}
    \item \textbf{GNU Multiple Precision Arithmetic Library (GMP)}: Used for efficient handling of arbitrary-precision integers, which is essential for working with numbers up to 4096 bits.
    
    \item \textbf{Standard C++ Libraries}: Used for various general-purpose functionality, including file I/O, timing, and container data structures.
\end{itemize}

\subsection{System Architecture}

The overall architecture of the system follows object-oriented design principles with clear interfaces between components:

\begin{figure}[H]
    \centering
    \begin{tikzpicture}
        % Define the nodes
        \node[draw, rectangle, rounded corners, minimum width=2.5cm, minimum height=1cm] (app) {Application};
        
        \node[draw, rectangle, rounded corners, minimum width=2.5cm, minimum height=1cm, below left=1cm of app] (prng) {PRNG};
        \node[draw, rectangle, rounded corners, minimum width=2.5cm, minimum height=1cm, below right=1cm of app] (primality) {Primality Tests};
        
        \node[draw, rectangle, rounded corners, minimum width=2.5cm, minimum height=1cm, below=1cm of prng] (lcg) {LCG};
        \node[draw, rectangle, rounded corners, minimum width=2.5cm, minimum height=1cm, right=0.5cm of lcg] (xoshiro) {Xoshiro256++};
        
        \node[draw, rectangle, rounded corners, minimum width=2.5cm, minimum height=1cm, below=1cm of primality] (miller) {Miller-Rabin};
        \node[draw, rectangle, rounded corners, minimum width=2.5cm, minimum height=1cm, left=0.5cm of miller] (baillie) {Baillie-PSW};
        
        \node[draw, rectangle, rounded corners, minimum width=8cm, minimum height=1cm, below=1cm of xoshiro] (gmp) {GMP Library};
        
        % Define the edges
        \draw[-latex] (app) -- (prng);
        \draw[-latex] (app) -- (primality);
        
        \draw[-latex] (prng) -- (lcg);
        \draw[-latex] (prng) -- (xoshiro);
        
        \draw[-latex] (primality) -- (miller);
        \draw[-latex] (primality) -- (baillie);
        
        \draw[-latex] (lcg) -- (gmp);
        \draw[-latex] (xoshiro) -- (gmp);
        \draw[-latex] (miller) -- (gmp);
        \draw[-latex] (baillie) -- (gmp);
    \end{tikzpicture}
    \caption{System architecture diagram showing main components and relationships}
    \label{fig:architecture}
\end{figure}

\subsection{Experiment Framework}

A significant part of the project is the framework for measuring algorithm performance on both standard computers and RISC-V platforms. This framework includes:

\begin{itemize}
    \item \textbf{Timing Measurement Tools}: Programs that measure the execution time of various operations with high precision.
    
    \item \textbf{Energy Measurement Tools}: Programs and scripts for measuring energy consumption during algorithmic operations.
    
    \item \textbf{Configuration System}: JSON-based configuration files that allow for flexible experiment setup.
    
    \item \textbf{Analysis Scripts}: Python scripts for statistical analysis and visualization of experimental results.
\end{itemize}

\subsection{Build System}

The project uses a Makefile-based build system that:

\begin{itemize}
    \item Supports various build targets (debug, release, test, benchmark)
    \item Manages dependencies
    \item Provides a consistent interface for compiling and running the code
\end{itemize}

\subsection{Documentation}

The project is thoroughly documented using:

\begin{itemize}
    \item \textbf{Code Comments}: Inline documentation explaining algorithm implementations and design decisions
    
    \item \textbf{UML Diagrams}: PlantUML diagrams illustrating architecture and workflows
    
    \item \textbf{README Files}: Markdown files providing overview and usage instructions
    
    \item \textbf{API Documentation}: Detailed documentation of classes and functions
\end{itemize} 